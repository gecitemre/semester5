\documentclass{article}
\usepackage[dvipsnames]{xcolor}
\usepackage{graphicx}
\usepackage{pgfplots}
\usepackage{float}
\usepackage{graphics}
\usepackage{tabu}

\usepgfplotslibrary{fillbetween}
\pgfplotsset{compat=1.18}
\author{Emre Geçit}

\title{\includegraphics{9.4.png}\\ IE407 - Homework 1}

\begin{document}
\maketitle

\newpage
\section*{Question 1}
\paragraph*{a)}
As can be seen in Figure 1, the optimal solution to this problem is on the point (200, 50), where the line for the wood constraint and the cotton constraint intersect.
At this point, the profit is 3200\$.
\newcommand\xmax{400}
\begin{figure}[H]
    \centering
    \begin{tikzpicture}
        \begin{axis}[
                legend style={nodes={scale=0.8}},
                grid,
                grid style={densely dashed},
                axis line style={->},
                xlabel=$x$,
                ylabel=$y$,
                xmin=0, xmax=\xmax,
                ymin=0, ymax=400,
                axis lines=center,
                axis on top=true,
                domain=0:\xmax,
            ]
            \addplot [draw=red,thick,name path=wool_constraint] {150-x/2}; \addlegendentry{Wool constraint: $3x + 6y \leq 900$}
            \addplot [draw=blue,thick,name path=cotton_constraint] {300-5*x/4}; \addlegendentry{Cotton constraint: $5x + 4y \leq 1200$}
            \addplot [draw=green, thick, name path=profit] {(3200-12*x)/16}; \addlegendentry{Profit: $12x + 16y = 3200$}
            \path[name path=xaxis] (0,0) -- (\xmax,0);
            \addplot[cyan!20] fill between[of=xaxis and wool_constraint,soft clip={domain=0:200}];
            \addplot[cyan!20] fill between[of=xaxis and cotton_constraint,soft clip={domain=200:\xmax}];
        \end{axis}
    \end{tikzpicture}
    \caption{Graphical Solution for Question 1a}
\end{figure}

\paragraph*{b)}
If the profit for a sweatshirt is increased by 1 and changed to 17\$, the profit function becomes $12x + 17y$.
As can be seen in Figure 2, the optimal solution for this problem does not change.
\begin{figure}[H]
    \centering
    \begin{tikzpicture}
        \begin{axis}[
                legend style={nodes={scale=0.8}},
                grid,
                grid style={densely dashed},
                axis line style={->},
                xlabel=$x$,
                ylabel=$y$,
                xmin=0, xmax=\xmax,
                ymin=0, ymax=400,
                axis lines=center,
                axis on top=true,
                domain=0:\xmax,
            ]
            \addplot [draw=red,thick,name path=wool_constraint] {150-x/2}; \addlegendentry{Wool constraint: $3x + 6y \leq 900$}
            \addplot [draw=blue,thick,name path=cotton_constraint] {300-5*x/4}; \addlegendentry{Cotton constraint: $5x + 4y \leq 1200$}
            \addplot [draw=green, thick, name path=profit] {(3200-12*x)/17}; \addlegendentry{Profit: $12x + 17y = 3200$}
            \path[name path=xaxis] (0,0) -- (\xmax,0);
            \addplot[cyan!20] fill between[of=xaxis and wool_constraint,soft clip={domain=0:200}];
            \addplot[cyan!20] fill between[of=xaxis and cotton_constraint,soft clip={domain=200:\xmax}];
        \end{axis}
    \end{tikzpicture}
    \caption{Graphical Solution for Question 1b}
\end{figure}

\paragraph*{c)}
When the profit for a t-shirt is changed to 20\$, the profit function becomes $20x + 16y$. In this case, the profit line's slope is the same as the cotton constraint line.
The optimal solution is all the points on the cotton constraint line, which satisfy the wool constraint.

If the profit is increased more, the wool constraint stops being a binding constraint, and the optimal solution changes to (240, 0).

\begin{figure}[H]
    \centering
    \begin{tikzpicture}
        \begin{axis}[
                legend style={nodes={scale=0.8}},
                grid,
                grid style={densely dashed},
                axis line style={->},
                xlabel=$x$,
                ylabel=$y$,
                xmin=0, xmax=\xmax,
                ymin=0, ymax=400,
                axis lines=center,
                axis on top=true,
                domain=0:\xmax,
            ]
            \addplot [draw=red,thick,name path=wool_constraint] {150-x/2}; \addlegendentry{Wool constraint: $3x + 6y \leq 900$}
            \addplot [draw=blue,thick,name path=cotton_constraint] {300-5*x/4}; \addlegendentry{Cotton constraint: $5x + 4y \leq 1200$}
            \addplot [draw=green, thick, name path=profit] {(4800-20*x)/16}; \addlegendentry{Profit: $20x + 16y = 4800$}
            \path[name path=xaxis] (0,0) -- (\xmax,0);
            \addplot[cyan!20] fill between[of=xaxis and wool_constraint,soft clip={domain=0:200}];
            \addplot[cyan!20] fill between[of=xaxis and cotton_constraint,soft clip={domain=200:\xmax}];
        \end{axis}
    \end{tikzpicture}
    \caption{Graphical Solution for Question 1c}
\end{figure}

\paragraph*{d)}
If 300 additional pounds of wool can be obtained, the wool constraint becomes $3x + 6y \leq 1200$. As can be seen in Figure 4, the optimal solution is on point (133.33, 133.33).
On this point, the profit is $12(133.33) + 16(133.33) = 3733.3$.

\begin{figure}[H]
    \centering
    \begin{tikzpicture}
        \begin{axis}[
                legend style={nodes={scale=0.8}},
                grid,
                grid style={densely dashed},
                axis line style={->},
                xlabel=$x$,
                ylabel=$y$,
                xmin=0, xmax=\xmax,
                ymin=0, ymax=400,
                axis lines=center,
                axis on top=true,
                domain=0:\xmax,
            ]
            \addplot [draw=red,thick,name path=wool_constraint] {200-x/2}; \addlegendentry{Wool constraint: $3x + 6y \leq 1200$}
            \addplot [draw=blue,thick,name path=cotton_constraint] {300-5*x/4}; \addlegendentry{Cotton constraint: $5x + 4y \leq 1200$}
            \addplot [draw=green, thick, name path=profit] {(3733.3-12*x)/16}; \addlegendentry{Profit: $12x + 16y = 3733.3$}
            \path[name path=xaxis] (0,0) -- (\xmax,0);
            \addplot[cyan!20] fill between[of=xaxis and wool_constraint,soft clip={domain=0:133.33333333}];
            \addplot[cyan!20] fill between[of=xaxis and cotton_constraint,soft clip={domain=133.33333333:\xmax}];
        \end{axis}
    \end{tikzpicture}
    \caption{Graphical Solution for Question 1d}
\end{figure}

\section*{Question 2}
\paragraph*{a)}
The products should be produced as follows in order to minimize the total cost:
\begin{table}[H]
    \centering
    \caption{Production amounts}
    \begin{tabular}{|l|l|l|l|l|}
    \hline
               & Handmade & Machine1 & Machine2 & Machine3 \\ \hline
    Cheesecake & 0        & 0        & 3000     & 0        \\ \hline
    Muffin     & 3000     & 0        & 0        & 2000     \\ \hline
    Cake       & 0        & 0        & 0        & 2000     \\ \hline
    \end{tabular}
    \end{table}
The minimized total cost is 32000\$.

\paragraph*{b)}
While the cost of producing cheesecake in machine 2 is in the range\\$[0, 4]$, the current basis remains optimal.

If the cost of producing cheesecake in machine 2 is increased by 1 unit, the current solution does not change. Minimized cost becomes 35000\$.

\paragraph*{c)}
While the cost of producing cake in machine 2 is in the range $[4, \infty)$, the current basis remains optimal.

If the cost of producing cake in machine 2 is increased by 1 unit, the current basis remains optimal. In this case, the minimized total cost is 32000\$.

If the cost of producing cake in machine 2 is decreased by two, the current basis no longer remains optimal. In this case, we need to solve the problem again.
While solved again, the products should be produced as follows to minimize the total cost:
\begin{table}[H]
    \centering
    \caption{Production amounts}
    \begin{tabular}{|l|l|l|l|l|}
    \hline
               & Handmade & Machine1 & Machine2 & Machine3 \\ \hline
    Cheesecake & 0        & 0        & 3000     & 0        \\ \hline
    Muffin     & 3000     & 0        & 0        & 2000     \\ \hline
    Cake       & 0        & 0        & 1500     & 500      \\ \hline
    \end{tabular}
    \end{table}
The minimized total cost is 30500\$.

\paragraph*{d)}
While the production requirement for muffins is in the range of $[0, 7500]$, the current basis remains optimal.

The shadow price for this constraint is 3\$.
If the production requirement for muffins is increased by 1 unit, the total cost increases by 3\$.

\paragraph*{e)}
While the time availability for machine 3 is in the range $[1000, \infty)$, the current basis remains optimal.

The shadow price for this constraint is 0\$.
If the time available for machine 3 is changed, the total cost does not change.

\paragraph*{f)}
If the model is forced to produce at least one handmade cheesecake, the problem should be solved again.
While solved again, the products should be produced as follows to minimize the total cost:
\begin{table}[H]
    \centering
    \caption{Production amounts}
    \begin{tabular}{|l|l|l|l|l|}
    \hline
               & Handmade & Machine1 & Machine2 & Machine3 \\ \hline
    Cheesecake & 1        & 0        & 2999     & 0        \\ \hline
    Muffin     & 2998.2     & 0        & 0        & 2001.8     \\ \hline
    Cake       & 0        & 0        & 2000     & 2000     \\ \hline
    \end{tabular}
\end{table}

In this case, the minimized total cost is 32002\$.

If we force the model to produce integer amounts of products, the solution becomes as follows:

\begin{table}[H]
    \centering
    \caption{Production amounts}
    \begin{tabular}{|l|l|l|l|l|}
    \hline
               & Handmade & Machine1 & Machine2 & Machine3 \\ \hline
    Cheesecake & 1        & 0        & 2999     & 0        \\ \hline
    Muffin     & 2999     & 0        & 0        & 2001     \\ \hline
    Cake       & 0        & 0        & 2000     & 2000     \\ \hline
\end{tabular}
\end{table}

In this case, the minimized total cost is again 32002\$.

While the cost of producing a handmade cheesecake is in the range of $[3, \infty)$, the current basis remains optimal.

If the cost of producing a handmade cheesecake becomes cheaper than the cost of producing a cheesecake in machine 2, producing handmade cheesecakes will be more beneficial.
Thus, if the cost of handmade cheesecakes falls below 3\$ per unit, handmade cheesecakes will be produced instead of cheesecakes in machine 2 and handmade cheesecakes will take a positive value.
This hypothesis can be verified by solving the problem again.

If we make the cost of producing a handmade cheesecake 2.99\$ per unit, the optimal solution becomes as follows:
\begin{table}[H]
    \centering
    \caption{Production amounts}
    \begin{tabular}{|l|l|l|l|l|}
    \hline
               & Handmade & Machine1 & Machine2 & Machine3 \\ \hline
    Cheesecake & 1111.111        & 0        & 1888.889     & 0        \\ \hline
    Muffin     & 1000     & 0        & 0        & 4000     \\ \hline
    Cake       & 0        & 0        & 0        & 2000     \\ \hline
    \end{tabular}
\end{table}

If we check the Sensitivity Report, we can see that the produced solution is valid in the range $[2.1, 3.0]$.

\begin{table}[H]
    \centering
    \caption{Sensitivity Report}
    \resizebox{\textwidth}{!}{%
    \begin{tabu}{l l l l l l}
        \hline
        \rowfont{\color{NavyBlue}\bfseries}
    Name & Value & Cost & Coefficient & Increase & Decrease \\ \hline
    Cheesecake Handmade & 1111.111111 & 0 & 2.99 & 0.01 & 0.89 \\ \hline
    \end{tabu}
    }
\end{table}%
When the cost of producing handmade cheesecakes falls below 2.1\$ per unit, the optimal solution changes.
Whatever the optimal solution becomes, the number of handmade cheesecakes definitely cannot be less than 1111.111, which is the number of handmade cheesecakes in the current solution; lowering the cost cannot result in a less amount of products.

It can be concluded that handmade cheesecakes are positive when the cost of producing a handmade cheesecake is less than 3\$ per unit.
\end{document}
